% $Header: /cvsroot/latex-beamer/latex-beamer/doc/beamercolorthemeexample.tex,v 1.11 2004/10/07 20:53:04 tantau Exp $
\documentclass{beamer}

\usepackage{beamerthemesplit}
\usepackage{graphicx,amsmath,natbib} %,apalike,cite
\usepackage{color}
\usepackage{subfigure}
\usepackage{bbm}
\usepackage{mathtools}
\usepackage{enumerate}
%\usepackage{graphicx,times,amsmath}

\DeclarePairedDelimiter\ceil{\lceil}{\rceil}
\DeclarePairedDelimiter\floor{\lfloor}{\rfloor}

\usetheme{Frankfurt}
\usecolortheme{seahorse}
\usecolortheme{rose}

\newcommand{\bfalpha} {\boldsymbol{\alpha}}
\newcommand{\bfbeta} {\boldsymbol{\beta}}
\newcommand{\bfgamma} {\boldsymbol{\gamma}}
\newcommand{\bfdelta} {\boldsymbol{\delta}}
\newcommand{\bfxi} {\boldsymbol{\xi}}
\newcommand{\bfmu} {\boldsymbol{\mu}}
\newcommand{\bfsigma} {\boldsymbol{\sigma}}
\newcommand{\bfeta} {\boldsymbol{\eta}}
\newcommand{\bfzeta} {\boldsymbol{\zeta}}
\newcommand{\bfvarphi} {\boldsymbol{\varphi}}
\newcommand{\bflambda} {\boldsymbol{\lambda}}
\newcommand{\bfpsi} {\boldsymbol{\psi}}
\newcommand{\bfphi} {\boldsymbol{\phi}}
\newcommand{\bfvarpsi} {\boldsymbol{\varpsi}}
\newcommand{\bfvarsigma} {\boldsymbol{\varsigma}}
\newcommand{\bfnu} {\boldsymbol{\nu}}
\newcommand{\bftheta} {\boldsymbol{\theta}}
\newcommand{\bfomega} {\boldsymbol{\omega}}
\newcommand{\bfTheta} {\boldsymbol{\Theta}}
\newcommand{\bfSigma} {\boldsymbol{\Sigma}}
\newcommand{\bfPsi} {\boldsymbol{\Psi}}
\newcommand{\bfchi} {\boldsymbol{\chi}}
\newcommand{\bfvartheta} {\boldsymbol{\vartheta}}

\newcommand{\bfx} {\mathbf{x}}
\newcommand{\bfy} {\mathbf{y}}
\newcommand{\bfY} {\mathbf{Y}}


\renewcommand{\Pr}{\mathsf{Pr}}
\newcommand{\E}{\mathsf{E}}
\newcommand{\V}{\mathsf{Var}}
\newcommand{\Cov}{\mathsf{Cov}}
\newcommand{\Cor}{\mathsf{Cor}}
\newcommand{\reals}{\mathbb{R}}

\DeclareMathOperator{\diag}{diag}
\DeclareMathOperator{\trace}{tr}
\newcommand{\normal}{\mathsf{N}}
\newcommand{\DP}{\mathsf{DP}}
\newcommand{\Ber}{\mathsf{Ber}}
\newcommand{\Dir}{\mathsf{Dir}}
\newcommand{\Gam}{\mathsf{G}}
\newcommand{\IGam}{\mathsf{IG}}
\newcommand{\bin}{\mathsf{Bin}}
\newcommand{\geo}{\mathsf{Geo}}
\newcommand{\Exp}{\mathsf{E}}
\newcommand{\Wis}{\mathsf{W}}
\newcommand{\IWis}{\mathsf{IW}}
\newcommand{\Poi}{\mathsf{Poi}}
\newcommand{\bet}{\mathsf{beta}}
\newcommand{\Mul}{\mathsf{Mult}}
\newcommand{\GDir}{\mathsf{GDir}}
\newcommand{\nDP}{\mathsf{nDP}}
\newcommand{\HDP}{\mathsf{HDP}}
\newcommand{\NIG}{\mathsf{NIG}}
\newcommand{\PDP}{\mathsf{PDP}}
\newcommand{\CRP}{\mathsf{CRP}}
\newcommand{\SB}{\mathsf{SB}}

\newcommand{\expo}[1]{ e^{ #1} }


\title{Various Methods for Estimating Volatility}
%\author{Georgi Dinolov, Abel Rodriguez - UC, Santa Cruz}
\date{5/4/2014}

\begin{document}

%\frame{\titlepage}

\begin{frame}
  \titlepage
\end{frame}

\begin{frame}
	\frametitle{The GARCH Process}
	The GARCH(p,q) process is given by 
	\begin{eqnarray*}
		\epsilon_t &=& \sigma_t z_t \\
		\sigma_t^2 &=& \alpha_0 + \sum_{i=1}^q \alpha_i \epsilon_{t-i}^2 + \sum_{i=1}^p \beta_i \sigma^2_{t-i},
	\end{eqnarray*}
	with $\epsilon_t$ being errors around a mean process, and $z_t$ is standard Brownian motion.

	The most common set of parameters are $p=1$ and $q=1$. Hansen and Lunde \cite{hansen2005comparison} show that GARCH(1,1) does as well, or better, than more complicated models in forecasting volatility. As a result, it is useful to focus on the properties of GARCH(1,1) as the time-step in the finite difference stochastic difference equation approaches zero.
\end{frame}


\begin{frame}
	\frametitle{Small-step asymptotics for GARCH(1,1)}
	If one is to use (regularly spaced) high-frequency data, it is useful to develop theory for ARCH processes where the time-step in the stochastic difference equation approaches zero. Nelson \cite{nelson1990arch-approximations} develops such asymptotic relations. Most notably, with respect to estimation of parameters, we expect to observe
	\begin{eqnarray*}
		\alpha_h + \beta_h &\to& 1 \mbox{ as } h \to 0 \\
		\alpha_h &\to& 0 \\
		\beta_h &\to& 1 
	\end{eqnarray*}

	\pause 
	
	Empirical evidence for high-frequency data does not support this.

	\pause

	What could cause theoretical deviations?
\end{frame}

\begin{frame}
	\frametitle{Reasons for GARCH failure at higher frequencies}
	\begin{enumerate}
		\item intraday volatility patterns 
		\item short-term bursts in volatility associated with news releases
		\item other microstructure noise
	\end{enumerate}

\pause

	The most commonly used method to model an intraday volatility process is Andersen and Bollerslev's sequential estimation approach \cite{andersen1997heterogeneous}, \cite{andersen1997intraday}
\end{frame}

\begin{frame}
	\frametitle{Realized Volatility}
	Assuming that logarithmic price of a financial asset is given by the diffusion process
	\begin{equation}
		p_t = \int_0^t \mu(s)ds + \int_0^t \sigma(s)dW(s),
	\end{equation}
	we are interested in the integrated volatility over a past time interval $\Delta$, 
	\begin{equation}
		IV_t = \int_{t-\Delta}^t \sigma^2(s)ds,
	\end{equation}
	
	\pause
	
	which is approximated with 
	
	\begin{equation}
		RV^{(m)} = \sum_{i=1}^m r_i^{(m)^2}.
	\end{equation}

	\pause

		From \cite{barndorff2002quadratic-variation}, $\frac{\sqrt{m} (RV^{(m)} - IV)}{\sqrt{2 \int_0^1 \sigma^4(s)ds}} \to N(0,1)$
\end{frame}

\begin{frame}
	\frametitle{Microstructure Noise}
	The very high-frequency prices are contaminated by market microstructure effects (noise), which lead to biases in realized volatility. Realized volatility estimators are derived based on the assumptions for the microscruture noise:
	\begin{equation}
		r_i^{(m)} = r_i^{*(m)} + \epsilon_i^{(m)}, i = 1,2, \ldots, m,
	\end{equation}
	where $r^{*(m)}_i$ is the true log-difference of prices, $\epsilon_i^{(m)}$ is the contamination noise, and $r^{(m)}$ is the observed log-difference price. 
\end{frame}

\begin{frame}
	\frametitle{RV decomposition in the presence of microstructure noise}
	\[ 
		RV^{(m)} = RV^{*(m)} + 2\sum_{i=1}^m r_i^{*(m)}\epsilon_i^{(m)} + \sum_{j=1}^m \epsilon_j^{(m)^2}
	\]
	Under the assumptions that $a)$ the noise process is independent and identically distributed with mean zero and finite variance $\omega^2$ and finite fourth moments, and $b)$ the noise is independent of the true price. 
	\begin{equation}
		E[ RV^{(m)} ] = IV + 2m\omega^2 
	\end{equation}

	\pause

	Sampling at lower frequencies reduces bias but leads to an increase in variance (bias-variance trade-off).
\end{frame}

\begin{frame}
	\frametitle{Averaging and Subsampling}
	Average a number of IV estimators for different sampling frequencies over the high-frequency samples:
	\begin{equation}
		TTSRV^{(m,m_1, \ldots, m_K, K)} = \frac{1}{K}\sum_{k=1}^K RV^{(k,m_k)} - \frac{\bar{m}}{m}RV^{(all)}
	\end{equation}
	Under independent noise assumptions, the estimator is consistent. Under equidistant observations and under regular allocation of the grids, there is an asymptotic distribution result. 
\end{frame}
	
\begin{frame}
	\frametitle{Kernel-Based Estimation}
	\begin{equation}
		KRV^{(m,H)} = RV^{(m)} + 2\sum_{h=1}^H \frac{m}{m-h}\gamma_h,
	\end{equation}
	with $\gamma_h = \sum_{i=1}^m r_i^{(m)} r_{i+h}^{(m)}$. 
\end{frame}

\begin{frame}
	\frametitle{Stochastic Volatility}
	We model the log-price $y^{*}(t)$ follows the solution to the stochastic differential equation
	\begin{equation}
		dy^{*}(t) = \{ \mu  + \beta\sigma^2(t) \}dt + \sigma(t)dw(t) 
	\end{equation}

	Here $\mu$ is the familiar drift and $\beta$ is the \textit{risk-premium}. Over an interval $\Delta > 0$ returns are defined as 
	\begin{eqnarray*}
		y_n &=& y^{*}(\Delta n) - y^{*}\{ (n-1)\Delta \} \\
		y_n | \sigma^2_n &\sim& N(\mu\Delta + \beta\sigma_n^2, \sigma^2_n ) \\
		\sigma_n^2 &=& \sigma^{2*} (n\Delta) - \sigma^{2*} \{ (n-1)\Delta \}
	\end{eqnarray*}

	\pause
	
	Models is subject to microstructure noise. We therefore fit the models to realized volatility over the trading periods $\Delta$. 
\end{frame}

\begin{frame}[allowframebreaks]
        \frametitle{References}
        \bibliographystyle{plain}
        \bibliography{C:/Advancement/advancement-bibliography}
\end{frame}


\end{document}